\documentclass[review]{elsarticle}

\usepackage{lineno,hyperref}
\usepackage{xcolor}
\modulolinenumbers[5]

\journal{Journal of Information Processing and Management }

\bibliographystyle{elsarticle-num}

\begin{document}

\begin{frontmatter}

\title{Investigating Facebook's actions against accounts that repeatedly share misinformation}

\author[mymainaddress]{Héloïse Théro\corref{mycorrespondingauthor}}
\cortext[mycorrespondingauthor]{Corresponding authors.}
\ead{thero.heloise@gmail.com}

\author[mymainaddress]{Emmanuel M. Vincent\corref{mycorrespondingauthor}}
\ead{emmanuel.vincent@sciencespo.fr}

\address[mymainaddress]{médialab - Sciences Po, Paris, France}

\begin{abstract}

{\color{red} SHORTEN AND ADD NEW CONDOR RESULTS}

Like many web platforms, Facebook is under pressure to regulate misinformation. 
According to the company, users that repeatedly share misinformation (`repeat offenders') will have their distribution reduced, but little is known about the implementation or the efficiency of this measure.
First, combining data from a fact-checking organization and CrowdTangle, we identified a set of public accounts (groups and pages) that have shared misinformation repeatedly. 
While we observe a decrease in engagement for pages (median of $-43\%$) after they shared two or more `false news', such a reduction is not observed for groups. 
However, we discover that groups have been affected in a different way with a sudden drop in their average engagement per post that occurred around June 9, 2020.
No public information was given by Facebook about this sudden decrease.
This drop has cut the groups’ engagement per post in half, but it was compensated by the fact that the overall activity of `repeat offenders' has doubled between 2019 and 2020.
Second, we identified pages that have been warned by Facebook and have shared a screenshot of the `reduced distribution' notification they have received. 
We found that their engagement per post following the notification decreased by a modest amount (median of $-24\%$), with some popular pages actually gaining more engagement.
Our results highlight easy steps Facebook could take to reduce misinformation, such as to enforce their `repeat offenders' policy more forcefully on pages, and to start applying it to groups.
\end{abstract}

\begin{keyword}
Misinformation\sep Content moderation\sep Algorithmic transparency\sep Facebook\sep Fact-checking\sep Social media analysis
%\MSC[2010] 00-01\sep  99-00
\end{keyword}

\end{frontmatter}

\linenumbers

\section{The Elsevier article class}

This is how you add a footnote.\footnote{This is the footnote message.}

Fully documented templates are available in the elsarticle package on \href{http://www.ctan.org/tex-archive/macros/latex/contrib/elsarticle}{CTAN}.

\paragraph{Installation} If the document class \emph{elsarticle} is not available on your computer, you can download and install the system package \emph{texlive-publishers} (Linux) or install the \LaTeX\ package \emph{elsarticle} using the package manager of your \TeX\ installation, which is typically \TeX\ Live or Mik\TeX.

\paragraph{Usage} Once the package is properly installed, you can use the document class \emph{elsarticle} to create a manuscript. Please make sure that your manuscript follows the guidelines in the Guide for Authors of the relevant journal. It is not necessary to typeset your manuscript in exactly the same way as an article, unless you are submitting to a camera-ready copy (CRC) journal.

\paragraph{Functionality} The Elsevier article class is based on the standard article class and supports almost all of the functionality of that class. In addition, it features commands and options to format the
\begin{itemize}
\item document style
\item baselineskip
\item front matter
\item keywords and MSC codes
\item theorems, definitions and proofs
\item labels of enumerations
\item citation style and labeling.
\end{itemize}

\section{Front matter}

The author names and affiliations could be formatted in two ways:
\begin{enumerate}[(1)]
\item Group the authors per affiliation.
\item Use footnotes to indicate the affiliations.
\end{enumerate}
See the front matter of this document for examples. You are recommended to conform your choice to the journal you are submitting to.

\subsection{Tables}
All tables should be numbered with Arabic numerals. Every table should have a caption. Headings should be placed above tables, left justified. Only horizontal lines should be used within a table, to distinguish the column headings from the body of the table, and immediately above and below the table. Tables must be embedded into the text and not supplied separately.

\subsection{Section headings}
Section headings should be left justified, bold, with the first letter capitalized and numbered consecutively, starting with the Introduction. Sub-section headings should be in capital and lower-case italic letters, numbered 1.1, 1.2, etc, and left justified, with second and subsequent lines indented. All headings should have a minimum of three text lines after them before a page or column break. Ensure the text area is not blank except for the last page.

\section{Figures and tables}
The figures should have captions at the bottom and the tables should have captions at the top.

\section{Bibliography styles}

There are various bibliography styles available. You can select the style of your choice in the preamble of this document. These styles are Elsevier styles based on standard styles like Harvard and Vancouver. Please use Bib\TeX\ to generate your bibliography and include DOIs whenever available.

Here are two sample references: \cite{Feynman1963118,Dirac1953888}.

\section{Acknowledgements}
Acknowledgements and Reference heading should be left justified, bold, with the first letter capitalized but have no numbers. Text below continues as normal.

\section*{References}

\bibliography{mybibfile}

\end{document}
